
\section{Estudio de Patrones}
\label{sec:patterns}


\begin{frame}{Calidad de los Datos}

  \begin{itemize} \justifying \parskip3mm
  \item Una de las partes más difícil y que más tiempo consume en la elaboración de pronósticos es la recopilación de datos válidos y confiables.
  \item Un pronóstico no puede ser tan preciso como los datos en que se basa. 
  \item El modelo de pronóstico más elaborado fallará si se aplica a datos poco confiables.
  \end{itemize}
 
\end{frame}

\begin{frame}{¿Cuáles Datos Son Útiles?}
  
  \begin{enumerate} \justifying \parskip5mm
  \item Los datos deben ser fidedignos y precisos.
  \item Los datos deberían ser relevantes.
  \item Los datos tienen que ser consistentes.
  \item Los datos deberían ser oportunos.
  \end{enumerate}

\end{frame}

\begin{frame}{Tipos De Datos}
  % 

  \begin{itemize}\justifying
  \item   Son dos los tipos de datos de interés para el pronosticador. El primer tipo son los datos recopilados en un periodo único, ya sea una hora, un día, una semana, un mes, o un trimestre. El segundo tipo son las observaciones de datos realizadas a través del tiempo. Cuando todas las observaciones se hacen durante el mismo periodo, las llamamos \alert{datos de corte transversal}.
  \item   Cualquier variable integrada con datos recopilados, registrados u observados durante
incrementos de tiempo sucesivos se llama \alert{serie de tiempo}. La producción mensual de cerveza
en Estados Unidos es un ejemplo de una \alert{serie de tiempo}.
  \end{itemize}
\end{frame}

\begin{frame}{Estudio de Patrones}{}
  Uno de los pasos más importantes en la selección de un método para pronosticar adecuado con datos de una serie de tiempo es considerar los diferentes \alert{tipos de patrones de datos}. Existen cuatro tipos generales: \alert{horizontal, tendencias, estacionales y cíclicos}.
\end{frame}



%%% Local Variables:
%%% mode: latex
%%% TeX-master: "slides"
%%% End:
